% file: chapters/tla.tex

\chapter{TLA+}	\label{chapter:tla}

\section{\text{UNCHANGED}}	\label{section:unchanged}

var1 == <<a, b>>

var2 == <<c, d>>

var == var1 \o var2

Error: ``a'' is either undefined or not an operator.

Solution: Use var == <<var1, var2>>.

\section{Model}	\label{section:model}

Question from hwayne on May 28, 2017:

Lamport defines a memory cache with the operator $NoVal \triangleq CHOOSE\; v : v \notin Val$.
He explicitly calls this best practice, because adding a `NOVAL` constant would be making the model more complex. 
Except unbounded CHOOSE is not checkable by TLC! 
While `NoVal` is nice for abstract specifications, 
it breaks things in the 99\% of cases where you're actually using TLA+.

Answer from pron on May 29, 2017:

You definitely should write $NoVal \triangleq CHOOSE\; v : v \notin Val$ in your spec 
(as that clarifies the intent) and override it as a "model value" (i.e., a constant) in TLC. 
This is best practice. 
Even $\exists x \in Nat$ is not checkable by TLC, 
yet best practice is to leave it like that (as that is the intent), 
and override `Nat` in TLC. 
In either case, the more general specification is verifiable in TLAPS.
